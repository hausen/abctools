%\def\numprova{A01}
% Se você quer dar um número para cada prova, descomente a
% linha acima
% se quiser totalizar o número de questões e de pontos,
% use a opção addpoints
\documentclass[addpoints,11pt]{provaufabc}
\usepackage[utf8]{inputenc}
\usepackage{enumerate}
\usepackage{amssymb,amsmath}

% Nome da prova
\title{Primeira Prova da Disciplina Exemplo}

% data da prova
\date{\today}

% autor
\author{Professor Rodrigo Hausen}

\begin{document}
% O vspace abaixo é só para economizar espaço na primeira
% folha. Você pode removê-lo.
\vspace*{-1cm}

% coloca o cabeçalho na folha de questões
\makeheader

% use o environment instructions para colocar as instruções
% para a prova
\begin{instructions}
% os labels totalquestions e totalpoints indicam,
% respectivamente, o número total de questões e pontos
\item A prova possui \ref{totalquestions} questões, com um
      total de \ref{totalpoints} pontos.
\item Escreva o seu nome e RA no espaço acima.
\item Após terminar a prova, entregue \textbf{todas} as folhas
      que você tiver, incluindo a folha de questões, o caderno
      de respostas e quaisquer folhas de rascunho que o
      professor tenha lhe fornecido.
\item Leia todas as questões antes de começar a fazer a prova.
\item As respostas podem ser escritas a lápis ou a caneta, como
      você achar melhor.
\item Não serão aceitas respostas sem justificativa.
\item Você pode resolver as questões na ordem que for melhor
      para você, mas mantenha a organização da prova. Deixe bem
      claro qual questão você está respondendo.
\item O verso desta folha pode ser usado para respostas (se
      você precisar de espaço) ou para rascunho. Se for usado
      para respostas, escreva ``respostas'' no verso.
\item Nem pense em colar; \textbf{cola = conceito F na
      disciplina}. Não se engane: \textbf{se você colar, você
      será descoberto!}
\end{instructions}

% use o environment cheatsheet para colocar dicas e lembretes
% de fórmulas
\begin{cheatsheet}
\sloppy
\item A equação do segundo grau $ax^2 + bx + c = 0$ tem no
      máximo duas soluções:
      $\displaystyle x_1 = \frac{-b - \sqrt{b^2 - 4ac}}{2a}$ e
      $\displaystyle x_2 = \frac{-b + \sqrt{b^2 - 4ac}}{2a}$.

% você pode dar um rótulo para 
\item Sejam $a, b$ os comprimentos dos catetos de um triângulo
      retângulo, e $c$ o comprimento de sua hipotenusa. Vale a
      identidade: $a^2 + b^2 = c^2$. Isto pode ser útil
      nas questões \ref{quest:triangulo}\ref{subquest:triangulo1}
      e \ref{quest:triangulo}\ref{subquest:triangulo2}.
\end{cheatsheet}

% use question para definir as questões
\begin{question}
\label{quest:triangulo}
    Para os triângulos abaixo, considere que $a$ e $b$ são os
    comprimentos dos catetos. Determine o comprimento da
    hipotenusa.

    % para uma questão dividida em várias partes, use o
    % environment subquestion.
    %
    % o environment subquestion recebe um parâmetro opcional,
    % o número de pontos da subquestão. Use o caractere ponto
    % "." como separador decimal.
    \begin{subquestion}[1.5]
        \label{subquest:triangulo1}
        $a = 1, b = 1$
    \end{subquestion}
    \begin{subquestion}[2.5]
        \label{subquest:triangulo2}
        $a = 3, b = 4$
    \end{subquestion}
\end{question}

% O environment question também pode receber o número de
% pontos da questão como parâmetro opcional. Se você
% definir o número de pontos na questão, não é permitido
% definir o número de pontos em uma subquestão.
\begin{question}[3.75]
Determine as raízes do polinômio $-x^2 + 2x + 1$.
\end{question}

\begin{question}[2.25]
Qual é a cor do cavalo branco de Napoleão?
\end{question}

\end{document}
